\section{Introduction}

This is the main text of the manuscript.
It should be written in LaTeX and can be split into multiple files.
For example, you could write files called ``01.introduction.tex'', ``02.methods.tex'', etc. and these would be automatically included in the final manuscript, in numerical order.
The big advantage of micromanubot is that you mustn't concern yourself with LaTeX boilerplate, manual image imports, or manual bibliography management.
Just write your content, locate figures using URLs (see below), cite references using DOIs \cite{@doi:10.1103/PhysRev.47.777}, then build the manuscript using ``umb build'', and a fully-formatted LaTeX manuscript and output will be generated for you.
Include figures using normal LaTeX commands.
For example, Figure \ref{fig:example} shows an example figure.

\begin{figure}[H]
    \centering
    \includegraphics[width=0.5\textwidth]{https://apastyle.apa.org/images/sample-figure-bar-graph_tcm11-261608_w1024_n.jpg}
    \caption{
        \textbf{Example Figure.} 
        This is an example figure.
    }
    \label{fig:example}
\end{figure}

Of course, local figures and citations can be used as well.
Simply provide figures in ``content/images/'' and citations \cite{umb} in \texttt{content/manual\_references.bib}.
